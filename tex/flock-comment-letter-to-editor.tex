\documentclass[11pt]{letter}
\usepackage{graphicx}


\textwidth = 6.5 in
\textheight = 10 in
\oddsidemargin = 0.0 in
\evensidemargin = 0.0 in
\topmargin = -.75 in
\headheight = 0.0 in
\headsep = 0.0 in
\parskip = 0.2in
\parindent = 0.0in

\date{}

\address{\mbox{}\vspace*{-1.0in}
\mbox{}\hfill
\includegraphics*{./images/NMFS_letterhead.pdf}\\} 

\signature{Eric C. Anderson \\ \mbox{} \\ eric.anderson@noaa.gov\\ ph.~831-420-3983}

\begin{document}

\begin{letter}{ \today 

Editor \\
Molecular Ecology Resources\\
}

\opening{Dear Editor:}
This letter accompanies our submission of, ``Interpreting the {\sc flock} algorithm from a statistical perspective,'' by Eric C. Anderson and Patrick D. Barry.  In this short comment we show that the method employed in the software {\sc flock} (published in {\em Molecular Ecology Resources} in 2009) can be identified as a special case of simulated annealing upon one of the models implemented in the software {\sc structure}.  

In two publications {\sc flock}'s authors have been somewhat outspoken in asserting that {\sc flock} is very different from {\sc structure}. Indeed, we embarked upon this project because numerous colleagues were recommending we try the program {\sc flock} in cases where {\sc structure} did not reveal much population structure.  Upon a close reading of the mathematical details of {\sc flock}'s algorithm, however, it was clear that {\sc flock} is very similar to {\sc structure}.  We describe this from a mathematical perspective and then perform analyses on a real data set to confirm it.  These analyses additionally expose some real problems that make it difficult to apply the {\sc flock} approach to estimating $K$, the number of clusters.  While we feel our findings would easily substantiate a position that there were few, if any, reasons to use {\sc flock} instead of {\sc structure}, we don't assert that point strongly in the paper. Rather, in consideration of the work that has been done to develop {\sc flock} and in the spirit of us all being molecular ecologists working together to understand the tools we use, we try to adopt a neutral tone, stating our mathematical and simulation findings as plainly as possible and leaving it to the reader to make up his or her own mind about which programs to use.  Of course, if the editors felt that a stronger stance were appropriate, we could provide that in a revised draft.


We suspect that {\sc flock} is being recommended to many others in our field as, ``very different from other clustering programs," which is something of a shame given that {\sc structure} is a firmly established and well-tested analytical tool in our field.  Consequently, we find our paper very relevant to the readership of {\em Molecular Ecology Resources}. 

A draft of this paper was sent to Julie Turgeon (the corresponding author on the original {\sc flock} paper) a little over one month ago, inviting her and her coauthor to provide comments and/or recommend any changes, but she has chosen to respond after submission.

We have given some thought to who would be good referees for this paper.  Given the technical and terse nature of our comparison between {\sc flock} and {\sc structure}, it would be advantageous to enlist referees who are very familiar with the model underlying {\sc structure}.  Obvious choices would be Jonathan Pritchard, Matthew Stephens, and John Novembre.  However, since these are very busy investigators running large human statistical-genetic labs, other possibilities would be Melissa Hubisz, Bruce Rannala, Greg Wilson (author of BayesAss), Philip Andolfatto, John Hulsenbeck, Jerome Pella, or Michelle Masuda.  Specifically in the molecular ecology realm, we think Oscar Gaggiotti would make a very good referee.

Daniel Falush would be the most appropriate editor for this manuscript because he has contributed to the development of {\sc structure} and is extremely familiar with the underlying model.  


\closing{Best regards,}
\setlength{\topmargin}{0in}
\textheight = 9 in
\end{letter}

 \end{document}   