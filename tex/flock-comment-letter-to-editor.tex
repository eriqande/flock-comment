\documentclass[11pt]{letter}
\usepackage{graphicx}


\textwidth = 6.5 in
\textheight = 10 in
\oddsidemargin = 0.0 in
\evensidemargin = 0.0 in
\topmargin = -.75 in
\headheight = 0.0 in
\headsep = 0.0 in
\parskip = 0.2in
\parindent = 0.0in

\date{}

\address{\mbox{}\vspace*{-1.0in}
\mbox{}\hfill
\includegraphics*{./images/NMFS_letterhead.pdf}\\} 

\signature{Eric C. Anderson \\ \mbox{} \\ eric.anderson@noaa.gov\\ ph.~831-420-3983}

\begin{document}

\begin{letter}{ \today 

Editor \\
Molecular Ecology Resources\\
}

\opening{Dear Editor:}
This letter accompanies our submission of, ``Interpreting the {\sc flock} algorithm from a statistical perspective,'' by Eric C. Anderson and Patrick D. Barry.  In this short comment we show that the method employed in the software {\sc flock} (published in {\em Molecular Ecology Resources} in 2009) can be identified as a special case of simulated annealing upon one of the models implemented in the software {\sc structure}.  

In two publications {\sc flock}'s authors have been somewhat outspoken in asserting that {\sc flock} is very different from {\sc structure}. Indeed, we embarked upon this project because numerous colleagues were recommending we try the program {\sc flock} in cases where {\sc structure} did not reveal much population structure.  Upon a close reading of the mathematical details of {\sc flock}'s algorithm, however, it was clear that {\sc flock} is very similar to {\sc structure}.  We describe this from a mathematical perspective and then perform analyses on a real data set to confirm it.  These analyses additionally expose some very real problems suffered by the {\sc flock} approach to estimating $K$, the number of clusters.  While we feel our findings would easily substantiate a position that there were few, if any, reasons to use {\sc flock} instead of {\sc structure}, we don't assert that point strongly in the paper. Rather, in consideration of the work that has been done to develop {\sc flock}, we try to adopt a neutral tone, stating our mathematical and simulation findings as plainly as possible and allowing readers to make up their own minds.  Of course, if the editors felt that a stronger stance were appropriate, we could provide that, as well.


We suspect that {\sc flock} is being recommended to many others in our field as, ``very different from other clustering programs," which is something of a shame given that {\sc structure} is a firmly established and well-tested analytical tool in our field.  Consequently, we find our paper very relevant to the readership of {\em Molecular Ecology Resources}. 

A draft of this paper was sent to Julie Turgeon (the corresponding author on the original {\sc flock} paper) a little over one month ago, inviting her and her coauthor to provide comments and/or recommend any changes, but she has chosen to respond after submission.


\closing{Best regards,}
\setlength{\topmargin}{0in}
\textheight = 9 in
\end{letter}

 \end{document}   