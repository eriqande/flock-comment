%!TEX root = flock-comment-main.tex

      We show that the the algorithm in the program {\sc flock} \citep{Duc&Tur2009} can be
interpreted as an estimation procedure based on 
a model essentially identical to the {\sc structure} 
\citep{Pritchardetal2000} model with no admixture and non-correlated 
allele frequency priors. Rather than using MCMC, the {\sc flock} algorithm 
searches for the maximum-{\em a-posteriori}
estimate of this {\sc structure} model via a simulated 
annealing algorithm with a rapid cooling 
schedule (namely, the exponent on the objective function~$\rightarrow \infty$).  We 
demonstrate the similarities between the two programs in a two step approach. First,
to enable rapid batch processing of many simulated data sets,
we modified the source code of  {\sc structure} to use the {\sc flock} algorithm, producing
the program {\sc flockture}. With simulated data we confirmed that results obtained with
{\sc flock}  and {\sc flockture} and very similar (though flockture is some 200 times faster). 
Second, we simulated multiple large datasets under varying 
levels of population differentiation for both microsatellite and SNP genotypes. We analyzed them
with {\sc flockture} and {\sc structure} and assessed each program on its ability to correctly cluster
individuals to their correct subpopulation.  We show that
{\sc flockture} yields results similar to {\sc structure} although with greater 
variability from run to run. {\sc flockture} did perform better than {\sc structure} 
when genotypes were composed of SNPs and differentiation was moderate 
($F_{ST}=0.022-0.032$). When differentiation was low, {\sc structure} outperformed {\sc flockture}
for both marker types. Interpreting {\sc flock}'s algorithm as a special case of the model in 
{\sc structure} should aid in understanding the program's output and behavior. 

