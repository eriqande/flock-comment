%!TEX root = flock-comment-main.tex

\section*{Introduction}
In most fields, proposed methods that are 
perceived to be fundamentally novel or new garner more
attention and accolades---and have a higher chance of publication---
than methods that are clearly minor elaborations upon existing work. 
Not surprisingly, then, in articles and manuscripts, authors may 
emphasize the differences and downplay the similarities between their 
work and the published literature.   In molecular ecology, this 
tendency amongst the creators of statistical methodology can make it 
difficult for end-users to understand the relationship between 
different methods.

Of course, new methods almost always build upon pre-existing ones, and 
much is to be gained by understanding the close relationship between 
different statistical methods in use in molecular ecology today.  
Indeed, some authors downplay the similarities between their method and 
existing ones not because they wish to make their method seem more 
novel, but rather because they are genuinely unaware of the close ties 
between their work and existing methods.  Helping to identify the 
similarities between new and existing work should help the field 
progress more quickly by 1) reducing end-user confusion about whether 
it is necessary to analyze data with a (sometimes overwhelming) variety 
of computer programs; 2) establishing a common language upon which to 
compare different methods; and 3) providing a principled perspective 
from which to argue about the expected strengths and weaknesses of any 
method.

